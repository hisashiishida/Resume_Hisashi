\documentclass{resume} % Use the custom resume.cls style

\usepackage[left=0.4 in,top=0.4in,right=0.4 in,bottom=0.4in]{geometry} % Document margins
\newcommand{\tab}[1]{\hspace{.2667\textwidth}\rlap{#1}} 
\newcommand{\itab}[1]{\hspace{0em}\rlap{#1}}
\name{Hisashi Ishida} % Your name
% You can merge both of these into a single line, if you do not have a website.
\address{{+1 667-910-2558} \\
\href{mailto:hishida3@jhu.edu}{hishida3@jhu.edu} \\ \href{https://www.linkedin.com/in/hisashi-ishida95}{www.linkedin.com/in/hisashi-ishida95}} 
\address{\\
\href{https://youtube.com/playlist?list=PLgAkMcuvrL6i8IJakd_I2R_A4B1KCt3tW}{https://youtube.com/playlist?list=PLgAkMcuvrL6i8IJakd_I2R_A4B1KCt3tW} 
% U.S. Citizen
}

\begin{document}

%----------------------------------------------------------------------------------------
%	OBJECTIVE
%----------------------------------------------------------------------------------------

% \begin{rSection}{OBJECTIVE}

% {I am a 2nd-year Ph.D. student at JHU with 4+ years of research experience, especially in surgical robotics, seeking an internship position in 2023 summer.}

% \end{rSection}
%----------------------------------------------------------------------------------------
%	EDUCATION SECTION
%----------------------------------------------------------------------------------------

\begin{rSection}{Education}

{\bf Johns Hopkins University, Baltimore, MD, USA}  \hfill {August 2021 - Present}\\
Ph.D. student in Computer Science
% Advisor: Professor Peter Kazanzides, Professor Russell Taylor
advised by Dr. Peter Kazanzides and  Dr. Russell H. Taylor

{\bf The University of Tokyo, Tokyo, Japan}  \hfill {April 2021 - August 2021}\\
Ph.D. student in Mechanical Engineering
% Advisor: Professor Harada Kanako
advised by Dr. Harada Kanako

{\bf The University of Tokyo, Tokyo, Japan}  \hfill {April 2019 - March 2021}\\
Master of Science in Mechanical Engineering
% Advisor: Professor Mamoru Mitsuishi
advised by Dr. Mamoru Mitsuishi

{\bf Massachusetts Institute of Technology, Massachusetts, U.S.A.}  \hfill {February 2017 - May 2017}\\
Special Student Program, Mechanical Engineering

{\bf The University of Tokyo, Tokyo, Japan}  \hfill {April 2015 - March 2019}\\
Bachelor of Science in Mechanical Engineering
% Advisor: Professor Mamoru Mitsuishi
advised by Dr. Mamoru Mitsuishi

\end{rSection}

%----------------------------------------------------------------------------------------
% TECHINICAL STRENGTHS	
%----------------------------------------------------------------------------------------

\begin{rSection}{EXPERIENCE}

\textbf{Engineer Intern} \hfill November 2020 - April 2021\\
SENSYN ROBOTICS \hfill \textit{Tokyo, JAPAN}
 \begin{itemize}
    \itemsep -3pt {} 
     \item Developed digital twin pipeline for remote control for power shovel to optimize human resources while enhancing construction site safety.
     \item Implemented the visual interface to monitor surrounding environment of power shovel and assist the operator by providing information of the target. 

     \item Skills: C++, Robot Operation System (ROS), RViz, Gazebo, Unified Robot Description Format (URDF)
 \end{itemize}
 
% \textbf{Engineer Intern} \hfill February 2020 - May 2020\\
% SCANX \hfill \textit{Tokyo, JAPAN}
%  \begin{itemize}
%     \itemsep -3pt {} 
%      \item Prototyped system for monitoring safety of employees in order to avoid construction site falls.
%      \item Developed system for tracking the employees' motion and tracing them on blueprints of the construction site.
%      \item Skills: C++, RealSense T265, ROS, RViz
%  \end{itemize}

\end{rSection} 

%----------------------------------------------------------------------------------------
%	WORK EXPERIENCE SECTION
%----------------------------------------------------------------------------------------

\begin{rSection}{PROJECTS}
\vspace{-1.25em}
\item \textbf{Digital Twin based assistance for improving situational awareness for skull-base Surgery.}\\
Johns Hopkins University, January 2023 - Present
\begin{itemize}
\itemsep -3pt {} 
\item Developed digital twin pipeline, which accurately represents the real-world robot motion and the preoperatively CT in the simulation within the millimeter accuracy.
\item Implemented SDF-based haptic feedback to avoid collision with critical structures.
\item Initial experiments using dental stone phantoms and cadaveric temporal bones demonstrate the system’s feasibility and effectiveness.
\end{itemize}

\item \textbf{SDF-based Guidance Modalities for
Mastoidectomy Procedures.}\\
Johns Hopkins University, January 2022 - May 2023
\begin{itemize}
\itemsep -3pt {} 
\item Developed a multimodal navigation system for a mastoidectomy VR simulation to identify the effect of different modalities (visual, audio, and haptic) on performance and mental demands.
    \item Implemented visual, audio, and haptic feedback using Signed Distance Field (SDF).
    \item Evaluated system with expert otolaryngology surgeons and shown that the system improved procedural safety without no additional time or workload.
\end{itemize}
\item \textbf{Semi-autonomous Assistance for Telesurgery
under Communication Loss.}\\
Johns Hopkins University, November 2021 - March 2023

\begin{itemize}
\itemsep -3pt {} 
    \item Proposed a telesurgery simulation framework that models an environment incorporating local and remote sites which can be applicable to provide high-quality surgery to medically underserved areas.
    \item Analyzed human behavior when there is a communication loss using the developed simulation and modeled the behavior using Kalman Filter.
    \item Provided different forms of assistance both under communication failure and when communication is restored. 
    
\end{itemize}
\item \textbf{Combined Segmentation Method for Harmonic scalpel using ResUnet and Classifier.}\\
Japan Society of Computer Aided Surgery AI Challenge, September 2020 - November 2020
\begin{itemize}
\itemsep -3pt {} 
\item Implemented semantic segmentation architecture (ResUnet) for the active blade of the harmonic scalpel and classification method for identifying the model of the scalpel.
    \item Adopted data augmentation methods to prevent over-fitting to dataset and improved accuracy to 90.6\%.
\end{itemize}
\item \textbf{Virtual Fixture Assistance for Suturing in Robot-Aided Pediatric Endoscopic Surgery.}\\
Master Thesis, the University of Tokyo, March 2021.
\begin{itemize}
\itemsep -3pt {} 
    \item Propose guidance virtual fixtures to enhance the performance and the safety of suturing while generating the required task constraints using constrained optimization and Cartesian force feedback. 
    \item Tested in simulations (CoppeliaSim) and experiments with a physical robot (DENSO VS050).
    \item Accepted to R-AL and presented at ICRA2020.
\end{itemize}
\end{rSection} 


%----------------------------------------------------------------------------------------
\begin{rSection}{Publication} 
\begin{itemize}
\item \textbf{Hisashi Ishida}*, Deepa Galaiya, Nimesh Nagururu,
Francis Creighton, Peter Kazanzides, Russell Taylor,
Manish Sahu*. “Beyond the Manual Touch: Situational-aware Force Control for Increased Safety in Robot-assisted Skullbase Surgery” Submitted to the 15th International Conference on Information Processing in
Computer-Assisted Interventions (IPCAI). Under Review (*equal contributions)


\item \textbf{Hisashi Ishida}*, Manish Sahu*, Adnan Munawar, Nimesh Nagururu, Deepa Galaiya, Peter Kazanzides, Francis X. Creighton, and Russell H. Taylor. “Haptic-Assisted Collaborative Robot Framework for Improved Situational Awareness in Skull Base Surgery” Submitted to 2024 2020 IEEE International Conference on Robotics and Automation (ICRA). Under Review (*equal contributions)

\item \textbf{Hisashi Ishida}*, Juan Antonio Barragan* \textit{et al.} “Improving Surgical Situational Awareness with Signed Distance Field: A Pilot Study in Virtual Reality” 2023 IEEE/RSJ International Conference on Intelligent Robots and Systems (IROS).(*equal contributions)

\item \textbf{Hisashi Ishida}, Adnan Munawar, Russell H. Taylor, Peter Kazanzides. “Semi-autonomous Assistance for Telesurgery under Communication Loss.” 2023 IEEE/RSJ International Conference on Intelligent Robots and Systems (IROS). 

\item \textbf{Hisashi Ishida}, Murilo M. Marinho, Kanako Harada, Mamoru Mitsuishi. “Preliminary Study on Looping Trajectory Classification for Robot-assisted Suturing in Pediatric Endoscopic Surgery.” 16th Asian Conference on Computer Aided Surgery (ACCAS), Tokyo, Japan, November 2020. Excellent Paper Award.

\item Risa Oikawa, Murilo M. Marinho, \textbf{Hisashi Ishida}, Kanako Harada, Mamoru Mitsuishi. “Towards the Semi-Automation of Looping in Robot Assisted Pediatric Endoscopic Surgery.” 16th Asian Conference on Computer Aided Surgery (ACCAS), Tokyo, Japan, November 2020.

\item Murilo M. Marinho*, \textbf{Hisashi Ishida}*, Kanako Harada, Kyoichi Deie, and Mamoru Mitsuishi. “Virtual Fixture Assistance for Suturing in Robot-Aided Pediatric Endoscopic Surgery.” IEEE Robotics and Automation Letters (RA-L), 5(2): 524–531, April 2020. Also presented at 2020 IEEE International Conference on Robotics and Automation (ICRA). (*equal contributions)

\item \textbf{Hisashi Ishida}, Murilo M. Marinho, Kanako Harada, Jian Gao, Mamoru Mitsuishi. “Virtual-Fixtures for Robotic-Assisted Bi-Manual Cutting Using Vector-Field Inequalities.” Proceedings of the 2020 IEEE/SICE International Symposium on System Integration (SII): 395-400, January 2020.

\item \textbf{Hisashi Ishida}, Murilo M. Marinho, Kanako Harada, Mamoru Mitsuishi. “Virtual Fixtures for Suturing in Robot-Aided Pediatric Endoscopic Surgery.” Proceedings of the 14th Asian Conference on Computer Aided Surgery (ACCAS): 50-51, November 2018.

\end{itemize}
\end{rSection}

\begin{rSection}{Awards and Honors}
\begin{itemize}
    \item \textbf{Honorable Mention} \hfill April 2022\\
    EN601.682 Machine Learning: Deep Learning course project
    Title: “Surgical Gesture Recognition in Videos and Kinematic Data.” Johns Hopkins University

    \item \textbf{Dean’s Award, The University of Tokyo} \hfill March 2021\\
    Awarded as the top graduate from Mechanical Engineering department.

    \item \textbf{Excellent Paper Award} \hfill November 2020\\
    16th Asian Conference on Computer Aided Surgery

    \item \textbf{Silver Award, Japan Society of Computer Aided Surgery AI Challenge} \hfill November 2020\\
    Surgical tool segmentation competition. 

    \item \textbf{Silver Award and Audience Award} \hfill March 2019\\
    Sony-University of Tokyo Startup Idea Competition

    \item \textbf{Best Design Award} (Mechanical Engineering Practice Modules 2 Course), \hfill March 2018\\
    Department of Mechanical Engineering, The University of Tokyo
    
\end{itemize}
\end{rSection}

%----------------------------------------------------------------------------------------
\begin{rSection}{SKILLS}
\begin{tabular}{ @{} >{}l @{\hspace{6ex}} l }
    \textbf{Programming skills:} C++, C, Python, OpenGL, OpenMP, MATLAB \\
    \textbf{Hardware:} Arudino, Raspberry Pi \\
    \textbf{Robot system design:} ROS, URDF  \\
    \textbf{Simulation:} AMBF, Rviz, CoppeliaSim, Gazebo \\
    \textbf{Deep learning framework:} PyTorch \\
\end{tabular}\\
\end{rSection}


\begin{rSection}{FELLOWSHIPS }
    \begin{itemize}
    \item \textbf{ITO Foundation fellowship} \hfill    2021 - 2023 \\
    Awarded 2-year fellowship (\$50,000/year) for Ph.D. study in the US. 


    \item \textbf{JSPS DC1 fellowship} \hfill   2021 \\
    Research fellowship(\$8,000 for research grant and \$2,000 for stipends) for young doctoral students from Japan Society for the Promotion of Science (JSPS).

    \item \textbf{Global Leader Program for Social Design and Management} \hfill 2019 - 2021\\
    Awarded monthly fellowship (\$2,000 for stipends) for graduate study, The University of Tokyo.

    \item \textbf{World-leading Innovative Graduate Study Program Co-Designing Future Society} \hfill 2019 - 2021\\
    Awarded monthly fellowship (\$2,000 for stipends) for graduate study, The University of Tokyo.
    
    \end{itemize}
\end{rSection}

\begin{rSection}{Others}
    \begin{itemize}
    \item Japanese: Native speaker
    \item English: Fluent speaker (TOEFL iBT 107).
    \end{itemize}
\end{rSection}

\end{document}
